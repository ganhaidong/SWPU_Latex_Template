
\documentclass[12pt,hyperref,a4paper,UTF8]{ctexart}
\usepackage{SWPUReport}


\usepackage{booktabs}   % 三线表
\usepackage{tabularx}   % 表格自动列宽
\usepackage{array}      % 更灵活的列格式(可选)
\usepackage[ruled,vlined,linesnumbered]{algorithm2e} % 伪代码

% =========================
%  让 \cite 变为右上角上标
%  (不引入 cite/natbib,避免和 .sty 里 hyperref 顺序冲突)
% =========================
\makeatletter
\renewcommand{\@cite}[2]{%
  \textsuperscript{[\,#1\if@tempswa ,\,#2\fi\,]}%
}
\makeatother

%%-------------------------------正文开始---------------------------%%
\begin{document}

%%-----------------------封面--------------------%%
\cover

\thispagestyle{empty} % 首页不显示页码

%%--------------------------目录页------------------------%%
\newpage
\tableofcontents

%%------------------------正文页从这里开始-------------------%
\newpage

\section{模板说明}
本模板主要适用于一些课程的平时论文以及期末论文,默认页边距为2.5cm,中文宋体,英文 Times New Roman,字号为12pt(小四)。

编译方式:\verb|xelatex -> bibtex -> xelatex*2|。

默认模板文件由以下四部分组成:
\begin{itemize}
    \item \texttt{main.tex} 主文件
    \item \texttt{reference.bib} 参考文献(BibTeX)
    \item \texttt{SWPUReport.sty} 文档格式控制(页眉、标题、学院、学号、姓名等)
    \item \texttt{figures} 放置图片的文件夹
\end{itemize}

第一次使用时需前往 \texttt{SWPUReport.sty} 对标题、姓名、学号、院所、页眉等进行设置。默认带有封面页以及目录页,页码从目录页开始。

\section{一些插入功能}

\subsection{插入公式}
行内公式 $v-\varepsilon+\phi=2$。

插入行间公式如\autoref{Euler}:
\begin{equation}
    v-\varepsilon+\phi=2
    \label{Euler}
\end{equation}

\subsection{插入图片}
SWPU 校徽如\autoref{swpu}所示。这里使用了 \verb|\autoref{}| 命令,会自动生成“图”“式”等前缀,无需手动输入。

此外,模板同时提供了校徽版本,如\autoref{swpu_notitle}所示。

\begin{figure}[!htbp]
    \centering
    \includegraphics[width=.5\textwidth]{figures/swpu_logo.pdf}
    \caption{西南石油大学}
    \label{swpu}
\end{figure}

\begin{figure}[!htbp]
    \centering
    \includegraphics[width=.4\textwidth]{figures/swpu_logo_notitle.pdf}
    \caption{校徽}
    \label{swpu_notitle}
\end{figure}

\subsection{插入文本框}
本模板定义了一个圆角灰底文本框,使用简化命令 \verb|\tbox{}| 即可(如不喜欢可在 \texttt{SWPUReport.sty} 中调整)。

\tbox{
    这是一个圆角灰底的文本框
}

\subsection{插入表格(三线表风格)}
顶会论文常用三线表(\texttt{booktabs}),原则上不建议使用竖线。示例如\autoref{tab:files}所示。

\begin{table}[!htbp]
    \centering
    \begin{tabular}{ll}
        \toprule
        文件名 & 说明 \\
        \midrule
        \texttt{main.tex}  & 主文件 \\
        \texttt{reference.bib} & 参考文献(BibTeX) \\
        \texttt{SWPUReport.sty}  & 文档格式控制 \\
        \texttt{figures}  & 图片文件夹 \\
        \bottomrule
    \end{tabular}
    \caption{本模板文件组成(三线表)}
    \label{tab:files}
\end{table}

% 若第二列内容较长,建议用 \texttt{tabularx} 自动换行(如\autoref{tab:filesx}):

% \begin{table}[!htbp]
%     \centering
%     \begin{tabularx}{0.9\textwidth}{lX}
%         \toprule
%         文件名 & 说明 \\
%         \midrule
%         \texttt{main.tex}  & 主文件 \\
%         \texttt{reference.bib} & 参考文献(BibTeX),支持 \verb|\cite| 引用 \\
%         \texttt{SWPUReport.sty}  & 文档格式控制(页眉、封面信息等)\\
%         \texttt{figures}  & 图片文件夹 \\
%         \bottomrule
%     \end{tabularx}
%     \caption{本模板文件组成(三线表,自动列宽)}
%     \label{tab:filesx}
% \end{table}

\subsection{插入伪代码}
课程作业常需写伪代码。这里使用 \texttt{algorithm2e},支持行号、输入输出、结构化控制语句等。示例如\autoref{alg:sgd}。

\begin{algorithm}[!htbp]
\caption{Mini-batch SGD 训练流程示例}
\label{alg:sgd}
\KwIn{数据集 $\mathcal{D}$,学习率 $\eta$,批大小 $B$,轮数 $T$}
\KwOut{模型参数 $\theta$}

初始化参数 $\theta$\;
\For{$t \leftarrow 1$ \KwTo $T$}{
    从 $\mathcal{D}$ 中采样一个 mini-batch $\mathcal{B}$,其中 $|\mathcal{B}|=B$\;
    计算梯度 $g \leftarrow \nabla_{\theta}\frac{1}{B}\sum_{(x,y)\in \mathcal{B}}\ell(f_{\theta}(x), y)$\;
    更新参数 $\theta \leftarrow \theta - \eta \cdot g$\;
}
\Return{$\theta$}\;
\end{algorithm}

\subsection{插入参考文献(上标引用)}
直接使用 \verb|\cite{}| 即可。本文已将引用设置为右上角上标形式。

例如:\textit{此处引用了文献\cite{OFDMAbackscatter}。此处引用了文献\cite{DigiScatter}。}

多篇引用也可连写:\textit{如\cite{OFDMAbackscatter,DigiScatter}所示。}

\section{写在最后}
\subsection{发布地址}
\begin{itemize}
    \item Github: \url{https://github.com/ganhaidong/SWPU_Latex_Template}
    \item Overleaf: \url{https://www.overleaf.com/gallery}
\end{itemize}

%%----------- 参考文献 -------------------%%
\reference

\end{document}
